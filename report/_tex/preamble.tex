%%
% Copyright (c) 2018, Pascal Wagler;  
% Copyright (c) 2014--2018, John MacFarlane
% 
% All rights reserved.
% 
% Redistribution and use in source and binary forms, with or without 
% modification, are permitted provided that the following conditions 
% are met:
% 
% - Redistributions of source code must retain the above copyright 
% notice, this list of conditions and the following disclaimer.
% 
% - Redistributions in binary form must reproduce the above copyright 
% notice, this list of conditions and the following disclaimer in the 
% documentation and/or other materials provided with the distribution.
% 
% - Neither the name of John MacFarlane nor the names of other 
% contributors may be used to endorse or promote products derived 
% from this software without specific prior written permission.
% 
% THIS SOFTWARE IS PROVIDED BY THE COPYRIGHT HOLDERS AND CONTRIBUTORS 
% "AS IS" AND ANY EXPRESS OR IMPLIED WARRANTIES, INCLUDING, BUT NOT 
% LIMITED TO, THE IMPLIED WARRANTIES OF MERCHANTABILITY AND FITNESS 
% FOR A PARTICULAR PURPOSE ARE DISCLAIMED. IN NO EVENT SHALL THE 
% COPYRIGHT OWNER OR CONTRIBUTORS BE LIABLE FOR ANY DIRECT, INDIRECT, 
% INCIDENTAL, SPECIAL, EXEMPLARY, OR CONSEQUENTIAL DAMAGES (INCLUDING,
% BUT NOT LIMITED TO, PROCUREMENT OF SUBSTITUTE GOODS OR SERVICES; 
% LOSS OF USE, DATA, OR PROFITS; OR BUSINESS INTERRUPTION) HOWEVER 
% CAUSED AND ON ANY THEORY OF LIABILITY, WHETHER IN CONTRACT, STRICT 
% LIABILITY, OR TORT (INCLUDING NEGLIGENCE OR OTHERWISE) ARISING IN 
% ANY WAY OUT OF THE USE OF THIS SOFTWARE, EVEN IF ADVISED OF THE 
% POSSIBILITY OF SUCH DAMAGE.
%%

%%
% For usage information and examples visit the GitHub page of this template:
% https://github.com/Wandmalfarbe/pandoc-latex-template
%%

\PassOptionsToPackage{unicode=true}{hyperref} % options for packages loaded elsewhere
\PassOptionsToPackage{hyphens}{url}
\PassOptionsToPackage{dvipsnames,svgnames*,table}{xcolor}
%
\documentclass[a4paper, listof=nochaptergap]{scrreprt}
\usepackage{lmodern}
\usepackage{setspace}

\setstretch{1.2}
\usepackage{amssymb,amsmath}
\usepackage{ifxetex,ifluatex}
\usepackage{fixltx2e} % provides \textsubscript
\ifnum 0\ifxetex 1\fi\ifluatex 1\fi=0 % if pdftex
  \usepackage[T1]{fontenc}
  \usepackage[utf8]{inputenc}
  \usepackage{textcomp} % provides euro and other symbols
\else % if luatex or xelatex
  \usepackage{unicode-math}
  \defaultfontfeatures{Ligatures=TeX,Scale=MatchLowercase}
\fi
% use upquote if available, for straight quotes in verbatim environments
\IfFileExists{upquote.sty}{\usepackage{upquote}}{}
% use microtype if available
\IfFileExists{microtype.sty}{%
\usepackage[]{microtype}
\UseMicrotypeSet[protrusion]{basicmath} % disable protrusion for tt fonts
}{}
\IfFileExists{parskip.sty}{%
\usepackage{parskip}
}{% else
\setlength{\parindent}{0pt}
\setlength{\parskip}{6pt plus 2pt minus 1pt}
}


\usepackage{ragged2e}
\usepackage{todonotes}

\usepackage{multicol}

\usepackage{subfig}
\usepackage[labelfont=bf, justification=justified]{caption}

%
% variables for title and author
%
\usepackage{titling}
\title{Design and Implementation of Computational Offloading in Mobile Edge Computing for Augmented Reality Applications}
\providecommand{\subtitle}[1]{}
\subtitle{Master thesis}
\author{Alex Justesen Karlsen}
\date{\today}

\usepackage{hyperref}
\hypersetup{
            pdftitle={\thetitle{}},
            pdfauthor={\theauthor},
            pdfsubject={RD Project},
            pdfkeywords={Markdown, Example},
            pdfborder={0 0 0},
            breaklinks=true}
\urlstyle{same}  % don't use monospace font for urls
\usepackage[margin=2.5cm,includehead=true,includefoot=true,centering]{geometry}
\usepackage{longtable,booktabs}
\usepackage{longtable,tabu}
\usepackage{multirow}


\usepackage{tocbasic}
%\addtotoclist[master-thesis.cls]{toc}
\renewcommand*{\tableofcontents}{\listoftoc[{\contentsname}]{toc}}% ToC under control of tocbasic
\AfterTOCHead[toc]{\addcontentsline{toc}{chapter}{ Contents}\thispagestyle{fancy}\pagestyle{fancy}}
\AfterStartingTOC[toc]{\clearpage}


\usepackage{ifplatform}


\ifwindows

\makeatletter
\def\tabu@verticalmeasure{\everypar{}%
	\unless\ifnum\currentgrouptype=14 \let\tabu@currentgrouptype\currentgrouptype\fi
	\ifnum \tabu@currentgrouptype>12         % 14=semi-simple, 15=math shift group
	\setbox\tabu@box =\hbox\bgroup
	\let\tabu@verticalspacing \tabu@verticalsp@lcr
	\d@llarbegin                % after \hbox ...
	\else
	\edef\tabu@temp{\ifnum\tabu@currentgrouptype=5\vtop
		\else\ifnum\tabu@currentgrouptype=12\vcenter
		\else\vbox\fi\fi}%
	\setbox\tabu@box \hbox\bgroup$\tabu@temp \bgroup
	\let\tabu@verticalspacing \tabu@verticalsp@pmb
	\fi
}
\makeatother

\fi

\usepackage[export]{adjustbox}
\usepackage{graphicx}
\usepackage{listings}
\newcommand{\passthrough}[1]{#1}
\setlength{\emergencystretch}{3em}  % prevent overfull lines
\providecommand{\tightlist}{%
  \setlength{\itemsep}{0pt}\setlength{\parskip}{0pt}}
\setcounter{secnumdepth}{5}
% Redefines (sub)paragraphs to behave more like sections
\ifx\paragraph\undefined\else
\let\oldparagraph\paragraph
\renewcommand{\paragraph}[1]{\oldparagraph{#1}\mbox{}}
\fi
\ifx\subparagraph\undefined\else
\let\oldsubparagraph\subparagraph
\renewcommand{\subparagraph}[1]{\oldsubparagraph{#1}\mbox{}}
\fi

% Make use of float-package and set default placement for figures to H
\usepackage{float}
\floatplacement{figure}{H}



%%
%% added
%%

%
% No language specified? take American English.
%

\ifnum 0\ifxetex 1\fi\ifluatex 1\fi=0 % if pdftex
  \usepackage[shorthands=off,main=english]{babel}
  \usepackage{blindtext}
\else
    % See issue https://github.com/reutenauer/polyglossia/issues/127
  \renewcommand*\familydefault{\sfdefault}
    % load polyglossia as late as possible as it *could* call bidi if RTL lang (e.g. Hebrew or Arabic)
  \usepackage{polyglossia}
  \setmainlanguage[]{english}
\fi

%
% Bibliography
%
\usepackage[backend=bibtex,style=ieee,natbib=true, citestyle=numeric-comp]{biblatex} %added
\addbibresource{_tex/biblio.bib}
%\usepackage[numbers]{natbib}
%\renewcommand{\bibsection}{}
%\bibliographystyle{ieee}


%
% Glossary and abbreviations
%
\usepackage[acronym]{glossaries}
\usepackage{glossary-mcols}


\usepackage{xparse}
\DeclareDocumentCommand{\newdualentry}{ O{} O{} m m m m } {
	\newglossaryentry{gls-#3}{name={#5},text={#5\glsadd{#3}},
		description={#6},#1
	}
	\makenoidxglossaries
	\newacronym[see={[Glossary:]{gls-#3}},#2]{#3}{#4}{#5\glsadd{gls-#3}}
}

\loadglsentries{_tex/glossary}

%
% colors
%
\usepackage[]{xcolor}

%
% listing colors
%
% \definecolor{listing-background}{HTML}{F7F7F7}
% \definecolor{listing-rule}{HTML}{B3B2B3}
% \definecolor{listing-numbers}{HTML}{B3B2B3}
% \definecolor{listing-text-color}{HTML}{000000}
% \definecolor{listing-keyword}{HTML}{435489}
% \definecolor{listing-identifier}{HTML}{435489}
% \definecolor{listing-string}{HTML}{00999A}
% \definecolor{listing-comment}{HTML}{8E8E8E}
% \definecolor{listing-javadoc-comment}{HTML}{006CA9}

%\definecolor{listing-background}{rgb}{0.97,0.97,0.97}
%\definecolor{listing-rule}{HTML}{B3B2B3}
%\definecolor{listing-numbers}{HTML}{B3B2B3}
%\definecolor{listing-text-color}{HTML}{000000}
%\definecolor{listing-keyword}{HTML}{D8006B}
%\definecolor{listing-identifier}{HTML}{000000}
%\definecolor{listing-string}{HTML}{006CA9}
%\definecolor{listing-comment}{rgb}{0.25,0.5,0.35}
%\definecolor{listing-javadoc-comment}{HTML}{006CA9}

%
% for the background color of the title page
%
\usepackage{pagecolor}
\usepackage{afterpage}

%
% TOC depth and 
% section numbering depth
%
\setcounter{tocdepth}{3}
\setcounter{secnumdepth}{3}

%
% break urls
%
\PassOptionsToPackage{hyphens}{url}

%
% When using babel or polyglossia with biblatex, loading csquotes is recommended 
% to ensure that quoted texts are typeset according to the rules of your main language.
%
\usepackage{csquotes}

%
% captions
%
\definecolor{caption-color}{HTML}{777777}
\usepackage[font={stretch=1.2}, textfont={color=caption-color}, position=top, skip=4mm, labelfont=bf, singlelinecheck=false, justification=raggedright]{caption}
\setcapindent{0em}
\captionsetup[longtable]{position=above}

\definecolor{main-color}{HTML}{000000}
%
% blockquote
%
\definecolor{blockquote-border}{RGB}{221,221,221}
\definecolor{blockquote-text}{RGB}{119,119,119}
\usepackage{mdframed}
\newmdenv[rightline=false,bottomline=false,topline=false,linewidth=3pt,linecolor=blockquote-border,skipabove=\parskip]{customblockquote}
\renewenvironment{quote}{\begin{customblockquote}\list{}{\rightmargin=0em\leftmargin=0em}%
\item\relax\color{blockquote-text}\ignorespaces}{\unskip\unskip\endlist\end{customblockquote}}

%
% Source Sans Pro as the de­fault font fam­ily
% Source Code Pro for monospace text
%
% 'default' option sets the default 
% font family to Source Sans Pro, not \sfdefault.
%
\usepackage[default]{sourcesanspro}
\usepackage{sourcecodepro}

%
% heading color
%
\definecolor{heading-color}{RGB}{40,40,40}
\addtokomafont{section}{\color{heading-color}}
% When using the classes report, scrreprt, book, 
% scrbook or memoir, uncomment the following line.
%\addtokomafont{chapter}{\color{heading-color}}
%\RedeclareSectionCommand[beforeskip=0pt,
%afterskip=10pt]{chapter}



%
% tables
%

\definecolor{table-row-color}{HTML}{F5F5F5}
\definecolor{table-rule-color}{HTML}{999999}

%\arrayrulecolor{black!40}
\arrayrulecolor{table-rule-color}     % color of \toprule, \midrule, \bottomrule
%\setlength\heavyrulewidth{0.3ex}      % thickness of \toprule, \bottomrule
%\renewcommand{\arraystretch}{1.3}     % spacing (padding)
%
%% Reset rownum counter so that each table
%% starts with the same row colors.
%% https://tex.stackexchange.com/questions/170637/restarting-rowcolors
%\let\oldlongtable\longtable
%\let\endoldlongtable\endlongtable
%\renewenvironment{longtable}{
%\rowcolors{3}{}{table-row-color!100}  % row color
%\oldlongtable} {
%\endoldlongtable
%\global\rownum=0\relax}
%
%% Unfortunately the colored cells extend beyond the edge of the 
%% table because pandoc uses @-expressions (@{}) like so: 
%%
%% \begin{longtable}[]{@{}ll@{}}
%% \end{longtable}
%%
%% https://en.wikibooks.org/wiki/LaTeX/Tables#.40-expressions

%
% remove paragraph indention
%
\setlength{\parindent}{0pt}
\setlength{\parskip}{6pt plus 2pt minus 1pt}
\setlength{\emergencystretch}{3em}  % prevent overfull lines

%
%
% Listings
%
%

\definecolor{listing-background}{HTML}{F7F7F7}
\definecolor{listing-rule}{HTML}{B3B2B3}
\definecolor{listing-numbers}{HTML}{B3B2B3}
\definecolor{listing-text-color}{HTML}{000000}
\definecolor{listing-keyword}{HTML}{3333CC}
\definecolor{listing-identifier}{HTML}{435489}
\definecolor{listing-string}{HTML}{990000}
\definecolor{listing-comment}{HTML}{669933}

\lstdefinestyle{eisvogel_listing_style}{
  language         = python,
  numbers          = left,
  xleftmargin      = 2.7em,
  framexleftmargin = 2.5em,
  backgroundcolor  = \color{listing-background},
  basicstyle       = \color{listing-text-color}\small\ttfamily{}\linespread{1.15}, % print whole listing small
  breaklines       = true,
  frame            = single,
  framesep         = 0.6mm,
  rulecolor        = \color{listing-rule},
  frameround       = ffff,
  tabsize          = 4,
  numberstyle      = \color{listing-numbers},
  aboveskip        = 1.0em,
  belowcaptionskip = 1.0em,
  keywordstyle     = \color{listing-keyword}\bfseries,
  classoffset      = 0,
  sensitive        = true,
  identifierstyle  = \color{listing-identifier},
  commentstyle     = \color{listing-comment},
  morecomment      = [s][\color{listing-comment}]{/**}{*/},
  stringstyle      = \color{listing-string},
  showstringspaces = false,
  escapeinside     = {/*@}{@*/}, % Allow LaTeX inside these special comments
  literate         =
  {á}{{\'a}}1 {é}{{\'e}}1 {í}{{\'i}}1 {ó}{{\'o}}1 {ú}{{\'u}}1
  {Á}{{\'A}}1 {É}{{\'E}}1 {Í}{{\'I}}1 {Ó}{{\'O}}1 {Ú}{{\'U}}1
  {à}{{\`a}}1 {è}{{\'e}}1 {ì}{{\`i}}1 {ò}{{\`o}}1 {ù}{{\`u}}1
  {À}{{\`A}}1 {È}{{\'E}}1 {Ì}{{\`I}}1 {Ò}{{\`O}}1 {Ù}{{\`U}}1
  {ä}{{\"a}}1 {ë}{{\"e}}1 {ï}{{\"i}}1 {ö}{{\"o}}1 {ü}{{\"u}}1
  {Ä}{{\"A}}1 {Ë}{{\"E}}1 {Ï}{{\"I}}1 {Ö}{{\"O}}1 {Ü}{{\"U}}1
  {â}{{\^a}}1 {ê}{{\^e}}1 {î}{{\^i}}1 {ô}{{\^o}}1 {û}{{\^u}}1
  {Â}{{\^A}}1 {Ê}{{\^E}}1 {Î}{{\^I}}1 {Ô}{{\^O}}1 {Û}{{\^U}}1
  {œ}{{\oe}}1 {Œ}{{\OE}}1 {æ}{{\ae}}1 {Æ}{{\AE}}1 {ß}{{\ss}}1
  {ç}{{\c c}}1 {Ç}{{\c C}}1 {ø}{{\o}}1 {å}{{\r a}}1 {Å}{{\r A}}1
  {€}{{\EUR}}1 {£}{{\pounds}}1 {«}{{\guillemotleft}}1
  {»}{{\guillemotright}}1 {ñ}{{\~n}}1 {Ñ}{{\~N}}1 {¿}{{?`}}1
  {…}{{\ldots}}1 {≥}{{>=}}1 {≤}{{<=}}1 {„}{{\glqq}}1 {“}{{\grqq}}1
  {”}{{''}}1
}
\lstset{style=eisvogel_listing_style}


\lstdefinelanguage{JavaScript}{
  keywordstyle=\color{listing-keyword}\bfseries,
  %keywordstyle=[2]\color{js-method-color},
  %keywordstyle = [3]\color{js-type-color},
  ndkeywordstyle=\color{listing-keyword}\bfseries,
  identifierstyle=\color{listing-identifier},
  commentstyle=\color{listing-comment}\ttfamily,
  stringstyle=\color{listing-string}\ttfamily,
  keywords={new, true, false, catch, return, null, catch, switch, if, in, while, do, else, case, break},
  %keywords=[2]{createElement, append, methodName},
  %keywords = [3]{Name, Error, boolean},
  ndkeywords={class, export, throw, implements, import, this, var, function, typeof},
  sensitive=false,
  comment=[l]{//},
  morecomment=[s]{/*}{*/},
  morestring=[b]',
  morestring=[b]"
}

\lstdefinelanguage{XML}{
  morestring      = [b]",
  moredelim       = [s][\bfseries\color{listing-keyword}]{<}{\ },
  moredelim       = [s][\bfseries\color{listing-keyword}]{</}{>},
  moredelim       = [l][\bfseries\color{listing-keyword}]{/>},
  moredelim       = [l][\bfseries\color{listing-keyword}]{>},
  morecomment     = [s]{<?}{?>},
  morecomment     = [s]{<!--}{-->},
  commentstyle    = \color{listing-comment},
  stringstyle     = \color{listing-string},
  identifierstyle = \color{listing-identifier}
}



%
% header and footer
%
\usepackage{fancyhdr}
\pagestyle{fancy}
\fancyhead{}
\fancyfoot{}
\lhead[\today]{}
\chead[]{}
\rhead[\thesubtitle{}]{\today}
\lfoot[\thepage]{\theauthor}
\cfoot[]{}
\rfoot[\theauthor]{\thepage}
\renewcommand{\headrulewidth}{0.4pt}
\renewcommand{\footrulewidth}{0.4pt}

%%
%% end added
%%
\usepackage{appendix}
\makeatletter
\newcommand\chapassect{%
	\def\@chapter[##1]##2{\ifnum \c@secnumdepth >\m@ne
		\refstepcounter{chapter}%
		\typeout{\@chapapp\space\thechapter.}%
		\addcontentsline{toc}{section}%
		{\protect\numberline{\thechapter}##1}%
		\else
		\addcontentsline{toc}{section}{##1}%
		\fi
		\chaptermark{##1}%
		\addtocontents{lof}{\protect\addvspace{10\p@}}%
		\addtocontents{lot}{\protect\addvspace{10\p@}}%
		\if@twocolumn
		\@topnewpage[\@makechapterhead{##2}]%
		\else
		\@makechapterhead{##2}%
		\@afterheading
		\fi}%
}
\makeatother

