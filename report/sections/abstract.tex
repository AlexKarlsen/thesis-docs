
    \hypertarget{abstract}{%
    \chapter*{Abstract}\label{sec:abstract}}
    \textcolor{caption-color}{Emerging applications, such as \gls{ar}/\gls{vr}, autonomous driving, mission critical \gls{iot} applications, and others, require extreme low latency of \gls{ai} decision feedback. The conventional approach is sending the sensor data, e.g., images, to the central cloud or data center to perform advanced \gls{ml} algorithms and send the results, e.g., object detection, classification, back to the end mobile devices. The conventional cloud-centric \gls{ml} framework cannot fulfill the stringent requirements of these emerging applications. \gls{mec} is a new computing paradigm which brings the computing units from the core of the network to the network edge. \gls{mec} has many benefits such as lower communication latency, higher reliability and resiliency, better security and privacy, scalability and context-awareness and others. Pushing \gls{ai} to the Edge is also known as Edge \gls{ai} or \gls{ei}.}

	Early exiting predictions from deep neural network is a promising technique for intelligent applications at the edge of the network. Edge intelligence addresses the accuracy-latency trade-off and allow for designing applications able to meet time constraints and obtain reasonable accuray.  or energy. Early exiting framework such as BrachyNet allow for cascaded  execution of the DNN between end-devices and edge server to become distributed DNN.  